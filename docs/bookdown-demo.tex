\documentclass[]{book}
\usepackage{lmodern}
\usepackage{amssymb,amsmath}
\usepackage{ifxetex,ifluatex}
\usepackage{fixltx2e} % provides \textsubscript
\ifnum 0\ifxetex 1\fi\ifluatex 1\fi=0 % if pdftex
  \usepackage[T1]{fontenc}
  \usepackage[utf8]{inputenc}
\else % if luatex or xelatex
  \ifxetex
    \usepackage{mathspec}
  \else
    \usepackage{fontspec}
  \fi
  \defaultfontfeatures{Ligatures=TeX,Scale=MatchLowercase}
\fi
% use upquote if available, for straight quotes in verbatim environments
\IfFileExists{upquote.sty}{\usepackage{upquote}}{}
% use microtype if available
\IfFileExists{microtype.sty}{%
\usepackage[]{microtype}
\UseMicrotypeSet[protrusion]{basicmath} % disable protrusion for tt fonts
}{}
\PassOptionsToPackage{hyphens}{url} % url is loaded by hyperref
\usepackage[unicode=true]{hyperref}
\hypersetup{
            pdftitle={Basic Stats},
            pdfauthor={Bill Last Updated:},
            pdfborder={0 0 0},
            breaklinks=true}
\urlstyle{same}  % don't use monospace font for urls
\usepackage{natbib}
\bibliographystyle{apalike}
\usepackage{color}
\usepackage{fancyvrb}
\newcommand{\VerbBar}{|}
\newcommand{\VERB}{\Verb[commandchars=\\\{\}]}
\DefineVerbatimEnvironment{Highlighting}{Verbatim}{commandchars=\\\{\}}
% Add ',fontsize=\small' for more characters per line
\usepackage{framed}
\definecolor{shadecolor}{RGB}{248,248,248}
\newenvironment{Shaded}{\begin{snugshade}}{\end{snugshade}}
\newcommand{\KeywordTok}[1]{\textcolor[rgb]{0.13,0.29,0.53}{\textbf{#1}}}
\newcommand{\DataTypeTok}[1]{\textcolor[rgb]{0.13,0.29,0.53}{#1}}
\newcommand{\DecValTok}[1]{\textcolor[rgb]{0.00,0.00,0.81}{#1}}
\newcommand{\BaseNTok}[1]{\textcolor[rgb]{0.00,0.00,0.81}{#1}}
\newcommand{\FloatTok}[1]{\textcolor[rgb]{0.00,0.00,0.81}{#1}}
\newcommand{\ConstantTok}[1]{\textcolor[rgb]{0.00,0.00,0.00}{#1}}
\newcommand{\CharTok}[1]{\textcolor[rgb]{0.31,0.60,0.02}{#1}}
\newcommand{\SpecialCharTok}[1]{\textcolor[rgb]{0.00,0.00,0.00}{#1}}
\newcommand{\StringTok}[1]{\textcolor[rgb]{0.31,0.60,0.02}{#1}}
\newcommand{\VerbatimStringTok}[1]{\textcolor[rgb]{0.31,0.60,0.02}{#1}}
\newcommand{\SpecialStringTok}[1]{\textcolor[rgb]{0.31,0.60,0.02}{#1}}
\newcommand{\ImportTok}[1]{#1}
\newcommand{\CommentTok}[1]{\textcolor[rgb]{0.56,0.35,0.01}{\textit{#1}}}
\newcommand{\DocumentationTok}[1]{\textcolor[rgb]{0.56,0.35,0.01}{\textbf{\textit{#1}}}}
\newcommand{\AnnotationTok}[1]{\textcolor[rgb]{0.56,0.35,0.01}{\textbf{\textit{#1}}}}
\newcommand{\CommentVarTok}[1]{\textcolor[rgb]{0.56,0.35,0.01}{\textbf{\textit{#1}}}}
\newcommand{\OtherTok}[1]{\textcolor[rgb]{0.56,0.35,0.01}{#1}}
\newcommand{\FunctionTok}[1]{\textcolor[rgb]{0.00,0.00,0.00}{#1}}
\newcommand{\VariableTok}[1]{\textcolor[rgb]{0.00,0.00,0.00}{#1}}
\newcommand{\ControlFlowTok}[1]{\textcolor[rgb]{0.13,0.29,0.53}{\textbf{#1}}}
\newcommand{\OperatorTok}[1]{\textcolor[rgb]{0.81,0.36,0.00}{\textbf{#1}}}
\newcommand{\BuiltInTok}[1]{#1}
\newcommand{\ExtensionTok}[1]{#1}
\newcommand{\PreprocessorTok}[1]{\textcolor[rgb]{0.56,0.35,0.01}{\textit{#1}}}
\newcommand{\AttributeTok}[1]{\textcolor[rgb]{0.77,0.63,0.00}{#1}}
\newcommand{\RegionMarkerTok}[1]{#1}
\newcommand{\InformationTok}[1]{\textcolor[rgb]{0.56,0.35,0.01}{\textbf{\textit{#1}}}}
\newcommand{\WarningTok}[1]{\textcolor[rgb]{0.56,0.35,0.01}{\textbf{\textit{#1}}}}
\newcommand{\AlertTok}[1]{\textcolor[rgb]{0.94,0.16,0.16}{#1}}
\newcommand{\ErrorTok}[1]{\textcolor[rgb]{0.64,0.00,0.00}{\textbf{#1}}}
\newcommand{\NormalTok}[1]{#1}
\usepackage{longtable,booktabs}
% Fix footnotes in tables (requires footnote package)
\IfFileExists{footnote.sty}{\usepackage{footnote}\makesavenoteenv{long table}}{}
\usepackage{graphicx,grffile}
\makeatletter
\def\maxwidth{\ifdim\Gin@nat@width>\linewidth\linewidth\else\Gin@nat@width\fi}
\def\maxheight{\ifdim\Gin@nat@height>\textheight\textheight\else\Gin@nat@height\fi}
\makeatother
% Scale images if necessary, so that they will not overflow the page
% margins by default, and it is still possible to overwrite the defaults
% using explicit options in \includegraphics[width, height, ...]{}
\setkeys{Gin}{width=\maxwidth,height=\maxheight,keepaspectratio}
\IfFileExists{parskip.sty}{%
\usepackage{parskip}
}{% else
\setlength{\parindent}{0pt}
\setlength{\parskip}{6pt plus 2pt minus 1pt}
}
\setlength{\emergencystretch}{3em}  % prevent overfull lines
\providecommand{\tightlist}{%
  \setlength{\itemsep}{0pt}\setlength{\parskip}{0pt}}
\setcounter{secnumdepth}{5}
% Redefines (sub)paragraphs to behave more like sections
\ifx\paragraph\undefined\else
\let\oldparagraph\paragraph
\renewcommand{\paragraph}[1]{\oldparagraph{#1}\mbox{}}
\fi
\ifx\subparagraph\undefined\else
\let\oldsubparagraph\subparagraph
\renewcommand{\subparagraph}[1]{\oldsubparagraph{#1}\mbox{}}
\fi

% set default figure placement to htbp
\makeatletter
\def\fps@figure{htbp}
\makeatother

\usepackage{booktabs}
\usepackage{amsthm}
\makeatletter
\def\thm@space@setup{%
  \thm@preskip=8pt plus 2pt minus 4pt
  \thm@postskip=\thm@preskip
}
\makeatother

\title{Basic Stats}
\author{Bill Last Updated:}
\date{19 January, 2020}

\begin{document}
\maketitle

{
\setcounter{tocdepth}{1}
\tableofcontents
}
\chapter*{Preface: Motivation}\label{my-section}
\addcontentsline{toc}{chapter}{Preface: Motivation}

All the notes I have done here are about basic stats. While I have tried
my best, probably there are still some typos and errors. Please feel
free to let me know in case you find one. Thank you!

\chapter{MLE}\label{intro}

\section{Basic idea of MLE}\label{basic-idea-of-mle}

Suppose that we flip a coin, \(y_i=0\) for tails and \(y_i=1\) for
heads. If we get \(p\) heads from \(n\) trials, we can get the
proportion of heads is \(p/n\), which is the sample mean. If we do not
do any further calculation, this is our best guess.

Suppose that the true proablity is \(\rho\), then we can get:

\[
\mathbf{L}(y_i)=\begin{cases} \rho \;\;\:   y_i = 1 \\ 1-\rho \;\;\:  y_i = 0 \end{cases}
\] Thus, we can also write it as follows.
\[\mathbf{L}(y_i) = \rho^{y_i}(1-\rho)^{1-y_i}\]

Thus, we can get:

\[\prod \mathbf{L}(y_i|\rho)=\rho^{\sum y_i}(1-\rho)^{\sum(1-y_i)}\]
Further, we can get a log-transformed format.

\[log (\prod \mathbf{L}(y_i|\rho))=\sum y_i log \rho + \sum(1-y_i) log(1-\rho)\]

To maximize the log-function above, we can calculate the derivative with
respect to \(\rho\).
\[\frac{\partial log (\prod \mathbf{L}(y_i|\rho)) }{\partial \rho}=\sum y_i \frac{1}{\rho}-\sum(1-y_i) \frac{1}{1-\rho}\]
Set the derivative to zero and solve for \(\rho\), we can get

\[\sum y_i \frac{1}{\rho}-\sum(1-y_i) \frac{1}{1-\rho}=0\]
\[\Rightarrow (1-\rho)\sum y_i - \rho \sum(1-y_i) =0\]
\[\Rightarrow \sum y_i-\rho\sum y_i - n\rho +\rho\sum y_i =0\]
\[\Rightarrow \sum y_i - n\rho  =0\]
\[\Rightarrow \rho  = \frac{\sum y_i}{n}=\frac{p}{n}\] Thus, we can see
that the \(\rho\) maximizing the likelihood function is equal to the
sample mean.

\section{Coin flip example, probit, and
logit}\label{coin-flip-example-probit-and-logit}

In the example above, we are not really trying to estimate a lot of
regression coefficients. What we are doing actually is to calculate the
sample mean, or intercept in the regresion sense. What does it mean?
Let's use some data to explain it.

Suppose that we flip a coin 20 times and observe 8 heads. We can use the
R's glm function to esimate the \(\rho\). If the result is consistent
with what we did above, we should observe that the \(cdf\) of the
esimate of \(\beta_0\) (i.e., intercept) should be equal to
\(8/20=0.4\).

\begin{Shaded}
\begin{Highlighting}[]
\NormalTok{coins<-}\KeywordTok{c}\NormalTok{(}\KeywordTok{rep}\NormalTok{(}\DecValTok{1}\NormalTok{,}\DataTypeTok{times=}\DecValTok{8}\NormalTok{),}\KeywordTok{rep}\NormalTok{(}\DecValTok{0}\NormalTok{,}\DataTypeTok{times=}\DecValTok{12}\NormalTok{))}
\KeywordTok{table}\NormalTok{(coins)}
\end{Highlighting}
\end{Shaded}

\begin{verbatim}
## coins
##  0  1 
## 12  8
\end{verbatim}

\begin{Shaded}
\begin{Highlighting}[]
\NormalTok{coins<-}\KeywordTok{as.data.frame}\NormalTok{(coins)}
\end{Highlighting}
\end{Shaded}

\subsection{Probit}\label{probit}

\begin{Shaded}
\begin{Highlighting}[]
\NormalTok{probitresults <-}\StringTok{ }\KeywordTok{glm}\NormalTok{(coins }\OperatorTok{~}\StringTok{ }\DecValTok{1}\NormalTok{, }\DataTypeTok{family =} \KeywordTok{binomial}\NormalTok{(}\DataTypeTok{link =} \StringTok{"probit"}\NormalTok{), }\DataTypeTok{data =}\NormalTok{ coins)}
\NormalTok{probitresults}
\end{Highlighting}
\end{Shaded}

\begin{verbatim}
## 
## Call:  glm(formula = coins ~ 1, family = binomial(link = "probit"), 
##     data = coins)
## 
## Coefficients:
## (Intercept)  
##     -0.2533  
## 
## Degrees of Freedom: 19 Total (i.e. Null);  19 Residual
## Null Deviance:       26.92 
## Residual Deviance: 26.92     AIC: 28.92
\end{verbatim}

\begin{Shaded}
\begin{Highlighting}[]
\KeywordTok{pnorm}\NormalTok{(probitresults}\OperatorTok{$}\NormalTok{coefficients)}
\end{Highlighting}
\end{Shaded}

\begin{verbatim}
## (Intercept) 
##         0.4
\end{verbatim}

As we can see the intercept is \(-0.2533\), and thus
\(\Phi(-0.2533471)=0.4\)

\subsection{Logit}\label{logit}

We can also use logit link to calculate the intercept as well. Recall
that

\[p(y=1)=\frac{1}{1+e^{-(\beta_0+\beta_1x_1+...+\beta_nx_n)}}=\frac{e^{\beta_0+\beta_1x_1+...+\beta_nx_n}}{1+e^{\beta_0+\beta_1x_1+...+\beta_nx_n}}\]
Thus,

\[p(y=1)=\frac{e^{\beta_0}}{1+e^{\beta_0}}\]

\begin{Shaded}
\begin{Highlighting}[]
\NormalTok{logitresults <-}\StringTok{ }\KeywordTok{glm}\NormalTok{(coins }\OperatorTok{~}\StringTok{ }\DecValTok{1}\NormalTok{, }\DataTypeTok{family =} \KeywordTok{binomial}\NormalTok{(}\DataTypeTok{link =} \StringTok{"logit"}\NormalTok{), }\DataTypeTok{data =}\NormalTok{ coins)}
\NormalTok{logitresults}\OperatorTok{$}\NormalTok{coefficients}
\end{Highlighting}
\end{Shaded}

\begin{verbatim}
## (Intercept) 
##  -0.4054651
\end{verbatim}

\begin{Shaded}
\begin{Highlighting}[]
\KeywordTok{exp}\NormalTok{(logitresults}\OperatorTok{$}\NormalTok{coefficients)}\OperatorTok{/}\NormalTok{(}\DecValTok{1}\OperatorTok{+}\KeywordTok{exp}\NormalTok{(logitresults}\OperatorTok{$}\NormalTok{coefficients))}
\end{Highlighting}
\end{Shaded}

\begin{verbatim}
## (Intercept) 
##         0.4
\end{verbatim}

Note that, the defaul link for the binomial in the glm function in
logit.

\section{Further on logit}\label{further-on-logit}

The probablity of \(y=1\) is as follows:

\[p=p(y=1)=\frac{1}{1+e^{-(\beta_0+\beta_1x_1+...+\beta_nx_n)}}=\frac{e^{\beta_0+\beta_1x_1+...+\beta_nx_n}}{1+e^{\beta_0+\beta_1x_1+...+\beta_nx_n}}\]

Thus, the likelihood function is as follows:

\[L=\prod p^{y_i}(1-p)^{1-y_i}=\prod (\frac{1}{1+e^{-(\beta_0+\beta_1x_1+...+\beta_nx_n)}})^{y_i}(\frac{1}{1+e^{\beta_0+\beta_1x_1+...+\beta_nx_n}})^{1-y_i}\]

\[=\prod (1+e^{-(\beta_0+\beta_1x_1+...+\beta_nx_n)})^{-y_i}(1+e^{\beta_0+\beta_1x_1+...+\beta_nx_n})^{-(1-y_i)}\]

Thus, the log-likelihood is as follows:
\[logL=\sum (-y_i \cdot log(1+e^{-(\beta_0+\beta_1x_1+...+\beta_nx_n)})-(1-y_i)\cdot log(1+e^{\beta_0+\beta_1x_1+...+\beta_nx_n}))\]

Typically, optimisers minimize a function, so we use negative
log-likelihood as minimising that is equivalent to maximising the
log-likelihood or the likelihood itself.

\begin{Shaded}
\begin{Highlighting}[]
\CommentTok{#Source of R code: https://www.r-bloggers.com/logistic-regression/}

\NormalTok{mle.logreg =}\StringTok{ }\ControlFlowTok{function}\NormalTok{(fmla, data)}
\NormalTok{\{}
  \CommentTok{# Define the negative log likelihood function}
\NormalTok{  logl <-}\StringTok{ }\ControlFlowTok{function}\NormalTok{(theta,x,y)\{}
\NormalTok{    y <-}\StringTok{ }\NormalTok{y}
\NormalTok{    x <-}\StringTok{ }\KeywordTok{as.matrix}\NormalTok{(x)}
\NormalTok{    beta <-}\StringTok{ }\NormalTok{theta[}\DecValTok{1}\OperatorTok{:}\KeywordTok{ncol}\NormalTok{(x)]}
    
    \CommentTok{# Use the log-likelihood of the Bernouilli distribution, where p is}
    \CommentTok{# defined as the logistic transformation of a linear combination}
    \CommentTok{# of predictors, according to logit(p)=(x%*%beta)}
\NormalTok{    loglik <-}\StringTok{ }\KeywordTok{sum}\NormalTok{(}\OperatorTok{-}\NormalTok{y}\OperatorTok{*}\KeywordTok{log}\NormalTok{(}\DecValTok{1} \OperatorTok{+}\StringTok{ }\KeywordTok{exp}\NormalTok{(}\OperatorTok{-}\NormalTok{(x}\OperatorTok\NormalTok{beta))) }\OperatorTok{-}\StringTok{ }\NormalTok{(}\DecValTok{1}\OperatorTok{-}\NormalTok{y)}\OperatorTok{*}\KeywordTok{log}\NormalTok{(}\DecValTok{1} \OperatorTok{+}\StringTok{ }\KeywordTok{exp}\NormalTok{(x}\OperatorTok\NormalTok{beta)))}
    \KeywordTok{return}\NormalTok{(}\OperatorTok{-}\NormalTok{loglik)}
\NormalTok{  \}}
  
  \CommentTok{# Prepare the data}
\NormalTok{  outcome =}\StringTok{ }\KeywordTok{rownames}\NormalTok{(}\KeywordTok{attr}\NormalTok{(}\KeywordTok{terms}\NormalTok{(fmla),}\StringTok{"factors"}\NormalTok{))[}\DecValTok{1}\NormalTok{]}
\NormalTok{  dfrTmp =}\StringTok{ }\KeywordTok{model.frame}\NormalTok{(data)}
\NormalTok{  x =}\StringTok{ }\KeywordTok{as.matrix}\NormalTok{(}\KeywordTok{model.matrix}\NormalTok{(fmla, }\DataTypeTok{data=}\NormalTok{dfrTmp))}
\NormalTok{  y =}\StringTok{ }\KeywordTok{as.numeric}\NormalTok{(}\KeywordTok{as.matrix}\NormalTok{(data[,}\KeywordTok{match}\NormalTok{(outcome,}\KeywordTok{colnames}\NormalTok{(data))]))}
  
  \CommentTok{# Define initial values for the parameters}
\NormalTok{  theta.start =}\StringTok{ }\KeywordTok{rep}\NormalTok{(}\DecValTok{0}\NormalTok{,(}\KeywordTok{dim}\NormalTok{(x)[}\DecValTok{2}\NormalTok{]))}
  \KeywordTok{names}\NormalTok{(theta.start) =}\StringTok{ }\KeywordTok{colnames}\NormalTok{(x)}
  
  \CommentTok{# Calculate the maximum likelihood}
\NormalTok{  mle =}\StringTok{ }\KeywordTok{optim}\NormalTok{(theta.start,logl,}\DataTypeTok{x=}\NormalTok{x,}\DataTypeTok{y=}\NormalTok{y, }\DataTypeTok{method =} \StringTok{'BFGS'}\NormalTok{, }\DataTypeTok{hessian=}\NormalTok{T)}
\NormalTok{  out =}\StringTok{ }\KeywordTok{list}\NormalTok{(}\DataTypeTok{beta=}\NormalTok{mle}\OperatorTok{$}\NormalTok{par,}\DataTypeTok{vcov=}\KeywordTok{solve}\NormalTok{(mle}\OperatorTok{$}\NormalTok{hessian),}\DataTypeTok{ll=}\DecValTok{2}\OperatorTok{*}\NormalTok{mle}\OperatorTok{$}\NormalTok{value)}
\NormalTok{\}}
\end{Highlighting}
\end{Shaded}

\begin{Shaded}
\begin{Highlighting}[]
\NormalTok{mydata =}\StringTok{ }\KeywordTok{read.csv}\NormalTok{(}\KeywordTok{url}\NormalTok{(}\StringTok{'https://stats.idre.ucla.edu/stat/data/binary.csv'}\NormalTok{))}
\NormalTok{mylogit1 =}\StringTok{ }\KeywordTok{glm}\NormalTok{(admit}\OperatorTok{~}\NormalTok{gre}\OperatorTok{+}\NormalTok{gpa}\OperatorTok{+}\KeywordTok{as.factor}\NormalTok{(rank), }\DataTypeTok{family=}\NormalTok{binomial, }\DataTypeTok{data=}\NormalTok{mydata)}

\NormalTok{mydata}\OperatorTok{$}\NormalTok{rank =}\StringTok{ }\KeywordTok{factor}\NormalTok{(mydata}\OperatorTok{$}\NormalTok{rank) }\CommentTok{#Treat rank as a categorical variable}
\NormalTok{fmla =}\StringTok{ }\KeywordTok{as.formula}\NormalTok{(}\StringTok{"admit~gre+gpa+rank"}\NormalTok{) }\CommentTok{#Create model formula}
\NormalTok{mylogit2 =}\StringTok{ }\KeywordTok{mle.logreg}\NormalTok{(fmla, mydata) }\CommentTok{#Estimate coefficients}


 \KeywordTok{print}\NormalTok{(}\KeywordTok{cbind}\NormalTok{(}\KeywordTok{coef}\NormalTok{(mylogit1), mylogit2}\OperatorTok{$}\NormalTok{beta))}
\end{Highlighting}
\end{Shaded}

\begin{verbatim}
##                          [,1]         [,2]
## (Intercept)      -3.989979073 -3.772676422
## gre               0.002264426  0.001375522
## gpa               0.804037549  0.898201239
## as.factor(rank)2 -0.675442928 -0.675543009
## as.factor(rank)3 -1.340203916 -1.356554831
## as.factor(rank)4 -1.551463677 -1.563396035
\end{verbatim}

\section{References}\label{references}

\url{http://www.columbia.edu/~so33/SusDev/Lecture_9.pdf}

\chapter{Score, Gradient and
Jacobian}\label{score-gradient-and-jacobian}

\section{Score}\label{score}

The score is the gradient (the vector of partial derivatives) of
\(log L(\theta)\), with respect to an m-dimensional parameter vector
\(\theta\).

\[S(\theta) = \frac{\partial\ell}{\partial \theta}\] Typically, they use
\(\nabla\) to denote the partical derivative.

\[\nabla \ell\]

Such differentiation will generate a \(m \times 1\) row vector, which
indicates the sensitivity of the likelihood.

Quote from Steffen Lauritzen's slides: ``Generally the solution to this
equation must be calculated by iterative methods. One of the most common
methods is the Newton--Raphson method and this is based on successive
approximations to the solution, using Taylor's theorem to approximate
the equation.''

For instance, using logit link, we can get the first derivative of log
likelihood logistic regression as follows. We can not really find
\(\beta\) easily to make the equation to be 0.

\[\begin{aligned}
\frac{\partial \ell} {\partial \beta} 
&= \sum_{i=1}^{n}x_i^T[y_i-\frac{e^{\beta^Tx_i}}{1+e^{\beta^Tx_i}}] \\
&=\sum_{i=1}^{n} x_i^T[y_i-\hat{y_i}]
\end{aligned}\]

\section{Fisher scoring}\label{fisher-scoring}

{[}I will come back to this later.{]}

\url{https://www2.stat.duke.edu/courses/Fall00/sta216/handouts/diagnostics.pdf}

\url{https://stats.stackexchange.com/questions/176351/implement-fisher-scoring-for-linear-regression}

\section{Gradient and Jacobian}\label{gradient-and-jacobian}

\textbf{Remarks}: This part discusses gradient in a more general sense.

When \(f(x)\) is only in a single dimension space:

\(\mathbb{R}^n \rightarrow \mathbb{R}\)

\[\nabla f(x)=[\frac{\partial f}{\partial x_1},\frac{\partial f}{\partial x_2},...,\frac{\partial f}{\partial x_n}]\]
When \(f(x)\) is only in a m-dimension space (i.e., Jacobian):
\(\mathbb{R}^n \rightarrow \mathbb{R^m}\)

\[Jac(f)=\begin{bmatrix}
\frac{\partial f_1}{\partial x_1} & \frac{\partial f_1}{\partial x_2} & \frac{\partial f_1}{\partial x_3} & ... & \frac{\partial f_1}{\partial x_n}\\
\frac{\partial f_2}{\partial x_1} & \frac{\partial f_2}{\partial x_2} & \frac{\partial f_2}{\partial x_3} & ... & \frac{\partial f_2}{\partial x_n} \\
...\\
\frac{\partial f_m}{\partial x_1} & \frac{\partial f_m}{\partial x_2} & \frac{\partial f_n}{\partial x_3} & ... & \frac{\partial f_m}{\partial x_n}
\end{bmatrix}\]

For instance,

\(\mathbb{R}^n \rightarrow \mathbb{R}\):

\[f(x,y)=x^2+2y\]
\[\nabla f(x,y)=[\frac{\partial f}{\partial x},\frac{\partial f}{\partial y}]=[2x,2]\]
\(\mathbb{R}^n \rightarrow \mathbb{R^m}\)

\[f(x,y)=(x^2+2y,x^3)\] \[Jac(f)=\begin{bmatrix}
2x & 2\\
2x^2 & 0 
\end{bmatrix}\]

\section{Hessian and Fisher
Information}\label{hessian-and-fisher-information}

Hessian matrix or Hessian is a square matrix of second-order partial
derivatives of a scalar-valued function, or scalar field.

\(\mathbb{R}^n \rightarrow \mathbb{R}\)

\[Hessian=\nabla ^2(f) =\begin{bmatrix}
\frac{\partial^2 f}{\partial x_1^2} & \frac{\partial^2 f}{\partial x_1 \partial x_2} & \frac{\partial^2 f}{\partial x_1 \partial x_3} & ... & \frac{\partial^2 f}{\partial x_1 \partial x_n}\\
\frac{\partial^2 f}{\partial x_2 \partial x_1} & \frac{\partial^2 f}{\partial x_2^2} & \frac{\partial^2 f}{\partial x_2 \partial x_3} & ... & \frac{\partial^2 f}{\partial x_2 \partial x_n} \\
\frac{\partial^2 f}{\partial x_3 \partial x_1} & \frac{\partial^2 f}{\partial x_3 \partial x_2} & \frac{\partial^2 f}{\partial x_3^2} & ... & \frac{\partial^2 f}{\partial x_3 \partial x_n} \\
...\\
\frac{\partial^2 f}{\partial x_n \partial x_1} & \frac{\partial^2 f}{\partial x_n \partial x_2} & \frac{\partial^2 f}{\partial x_n \partial x_3} & ... & \frac{\partial^2 f}{\partial x_n^2}
\end{bmatrix}\]

As a special case, in the context of logit:

Suppose that the log likelihood function is \(\ell (\theta)\).
\(\theta\) is a \(m\) demension vector.

\[ \theta = \begin{bmatrix}\theta_1 \\
\theta_2 \\
\theta_3 \\
\theta_4 \\
...\\
\theta_m \\
\end{bmatrix}\]

\[Hessian=\nabla ^2(\ell) =\begin{bmatrix}
\frac{\partial^2 \ell}{\partial \theta_1^2} & \frac{\partial^2 \ell}{\partial \theta_1 \partial \theta_2} & \frac{\partial^2 \ell}{\partial \theta_1 \partial \theta_3} & ... & \frac{\partial^2 \ell}{\partial \theta_1 \partial \theta_m}\\
\frac{\partial^2 \ell}{\partial \theta_2 \partial \theta_1} & \frac{\partial^2 \ell}{\partial \theta_2^2 } & \frac{\partial^2 \ell}{\partial \theta_1 \partial \theta_3} & ... & \frac{\partial^2 \ell}{\partial \theta_1 \partial \theta_m} \\
\frac{\partial^2 \ell}{\partial \theta_3 \partial \theta_1} & \frac{\partial^2 \ell}{\partial \theta_3 \theta_2 } & \frac{\partial^2 \ell}{\partial \theta_3^2} & ... & \frac{\partial^2 \ell}{\partial \theta_3 \partial \theta_m} \\
...\\
\frac{\partial^2 \ell}{\partial \theta_m \partial \theta_1} & \frac{\partial^2 \ell}{\partial \theta_m \theta_2 } & \frac{\partial^2 \ell}{\partial \theta_m \partial \theta_3} & ... & \frac{\partial^2 \ell}{\partial \theta_m \partial \theta_m} 
\end{bmatrix}\]

``In statistics, the observed information, or observed Fisher
information, is the negative of the second derivative (the Hessian
matrix) of the''log-likelihood" (the logarithm of the likelihood
function). It is a sample-based version of the Fisher information."
(Direct quote from Wikipedia.)

Thus, the observed information matrix:

\[-Hessian=-\nabla ^2(\ell) \]

Expected (Fisher) information matrix:

\[E[-\nabla ^2(\ell)] \]

\chapter{Canonical link function}\label{canonical-link-function}

Inspired by a Stack Exchange post, I created the following figure:

\[ \frac{Paramter}{\theta} \longrightarrow \gamma^{'}(\theta) = \mu \longrightarrow \frac{Mean}{\mu} \longrightarrow g(\mu) = \eta \longrightarrow \frac{ Linear predictor}{\eta} \]

For the case of \(n\) time Bernoulli (i.e., Binomial), its canonical
link function is logit. Specifically,

\[ \frac{Paramter}{\theta=\beta^Tx_i}  \longrightarrow \gamma^{'}(\theta)= \frac{e^{\beta^Tx_i}}{1+e^{\beta^Tx_i}}\longrightarrow \frac{Mean}{\mu=\frac{e^{\beta^Tx_i}}{1+e^{\beta^Tx_i}}}\longrightarrow g(\mu) = log \frac{\frac{e^{\beta^Tx_i}}{1+e^{\beta^Tx_i}}}{1-\frac{e^{\beta^Tx_i}}{1+e^{\beta^Tx_i}}}\longrightarrow \frac{ Linear predictor}{\eta = \beta^Tx_i}\]
Thus, we can see that,

\[\theta \equiv \eta \] The link function \(g(\mu)\) relates the linear
predictor \(\eta = \beta^Tx_i\) to the mean \(\mu\).

\textbf{Remarks}:

\begin{enumerate}
\def\labelenumi{(\arabic{enumi})}
\item
  Parameter is \(\theta = \beta ^T x_i\) (Not \(\mu\)!).
\item
  \(\mu=p(y=1)=\frac{e^{\beta^Tx_i}}{1+e^{\beta^Tx_i}}\) (Not logit!).
\item
  Link function (i.e., \(g(\mu)\)) = logit = logarithm of odds = log
  \(\frac{Event - Happened }{Event - Not - Happened}\).
\item
  \(g(\mu) = log \frac{\mu}{1-\mu}=\beta^T x_i\). Thus, link function =
  linear predictor = log odds!
\item
  Quote from the Stack Exchange post ``Newton Method and Fisher scoring
  for finding the ML estimator coincide, these links simplify the
  derivation of the MLE.''
\end{enumerate}

(Recall, we know that \(\mu\) or \(p(y=1)\) is the mean function. Recall
that, \(n\) trails of coin flips, and get \(p\) heads. Thus
\(\mu = \frac{p}{n}\).)

\chapter{Ordinary Least Squares (OLS)}\label{ordinary-least-squares-ols}

Suppose we have \(n\) observation, and \(m\) variables.

\[\begin{bmatrix}
x_{11} & x_{12} & x_{13} & ... & x_{1m}\\
x_{21} & x_{22} & x_{23} & ... & x_{2m} \\
...\\
x_{n1} & x_{n2} & x_{n3} & ... & x_{nm}
\end{bmatrix}\]

Thus, we can write it as the following \(n\) equations.

\[y_1=\beta_0+\beta_1 x_{11}+\beta_2 x_{12}+...+ \beta_m x_{1m}\]
\[y_2=\beta_0+\beta_1 x_{21}+\beta_2 x_{22}+...+ \beta_m x_{2m}\]
\[y_3=\beta_0+\beta_1 x_{31}+\beta_2 x_{32}+...+ \beta_m x_{3m}\]
\[...\]

\[y_n=\beta_0+\beta_1 x_{n1}+\beta_2 x_{n2}+...+ \beta_m x_{nm}\]

We can combine all the \(n\) equations as the following one:

\[y_i=\beta_0+\beta_1 x_{i1}+\beta_2 x_{i2}+...+ \beta_m x_{im}  (i \in [1,n])\]

We can further rewrite it as a matrix format as follows.

\[y= X \beta\] Where,

\[y = \begin{bmatrix}y_1 \\
y_2 \\
y_3 \\
y_4 \\
...\\
y_n \\
\end{bmatrix}\]

\[X=\begin{bmatrix}
1 & x_{11} & x_{12} & x_{13} & ... & x_{1m}\\
1 & x_{21} & x_{22} & x_{23} & ... & x_{2m} \\
...\\
1 & x_{n1} & x_{n2} & x_{n3} & ... & x_{nm}
\end{bmatrix}\]

\[\beta = \begin{bmatrix}\beta_0 \\
\beta_1 \\
\beta_2 \\
\beta_3 \\
...\\
\beta_m \\
\end{bmatrix}\]

Since later we need the inverse of \(X\), we need to make it into a
square matrix.

\[X^Ty=X^TX \hat{\beta} \Rightarrow \hat{\beta} = (X^TX)^{-1} X^Ty\]

We can use R to implement this calculation. As we can see, there is no
need to do any iterations at all, but rather just pure matrix
calculation.

\begin{Shaded}
\begin{Highlighting}[]
\NormalTok{X<-}\KeywordTok{matrix}\NormalTok{(}\KeywordTok{rnorm}\NormalTok{(}\DecValTok{1000}\NormalTok{),}\DataTypeTok{ncol=}\DecValTok{2}\NormalTok{) }\CommentTok{# we define a 2 column matrix, with 500 rows}
\NormalTok{X<-}\KeywordTok{cbind}\NormalTok{(}\DecValTok{1}\NormalTok{,X) }\CommentTok{# add a 1 constant}
\NormalTok{beta_true<-}\KeywordTok{c}\NormalTok{(}\DecValTok{2}\NormalTok{,}\DecValTok{1}\NormalTok{,}\DecValTok{2}\NormalTok{) }\CommentTok{# True regression coefficients}
\NormalTok{beta_true<-}\KeywordTok{as.matrix}\NormalTok{(beta_true)}
\NormalTok{y=X}\OperatorTok\NormalTok{beta_true}\OperatorTok{+}\KeywordTok{rnorm}\NormalTok{(}\DecValTok{500}\NormalTok{)}

\NormalTok{transposed_X<-}\KeywordTok{t}\NormalTok{(X)}
\NormalTok{beta_hat<-}\KeywordTok{solve}\NormalTok{(transposed_X}\OperatorTok\NormalTok{X)}\OperatorTok\NormalTok{transposed_X}\OperatorTok\NormalTok{y}
\NormalTok{beta_hat}
\end{Highlighting}
\end{Shaded}

\begin{verbatim}
##           [,1]
## [1,] 1.9981603
## [2,] 0.9808431
## [3,] 1.9731422
\end{verbatim}

\textbf{Side Notes} The function of as.matrix will automatically make
c(2,1,2) become the dimension of \(3 \times 1\), you do not need to
transpose the \(\beta\).

\section{Taylor series}\label{taylor-series}

\[\begin{aligned}
f(x)|_{a} &=f(a)+\frac{f^{'}(a)}{1!}(x-a)+\frac{f^{'}(a)}{2!}(x-a)^2+\frac{f^{''}(a)}{3!}(x-a)^{3}+...\\&=\sum_{n=0}^{\infty} \frac{f^{n}(a)}{n!}(x-a)^n 
\end{aligned}\]

For example:

\[\begin{aligned} 
e^x |_{a=0} &= e^a+ \frac{e^a}{1!}(x-a)+\frac{e^a}{2!}(x-a)^2+...+\frac{e^a}{n!}(x-a)^n \\ 
&=  1+ \frac{1}{1!}x+\frac{1}{2!}x^2+...+\frac{1}{n!}x^n
\end{aligned}\]

if \(x=2\)

\(e^2 = 7.389056\)

\(e^2 \approx 1+\frac{1}{1!}x =1+\frac{1}{1!}2=3\)

\(e^2 \approx 1+\frac{1}{1!}x+\frac{1}{2!}x^2 =1+\frac{1}{1!}2 + \frac{1}{2!}2 =5\)
\ldots{}

\(e^2 \approx 1+\frac{1}{1!}x+\frac{1}{2!}x^2 +\frac{1}{3!}x^2+\frac{1}{4!}x^2+\frac{1}{5!}x^2=7.2666...\)

\section{References}\label{references-1}

\begin{enumerate}
\def\labelenumi{\arabic{enumi}.}
\tightlist
\item
  Steffen Lauritzen's slides:
\end{enumerate}

\url{http://www.stats.ox.ac.uk/~steffen/teaching/bs2HT9/scoring.pdf}

\begin{enumerate}
\def\labelenumi{\arabic{enumi}.}
\setcounter{enumi}{1}
\tightlist
\item
  The Stack Exchange post:
\end{enumerate}

\url{https://stats.stackexchange.com/questions/40876/what-is-the-difference-between-a-link-function-and-a-canonical-link-function}

\begin{enumerate}
\def\labelenumi{\arabic{enumi}.}
\setcounter{enumi}{2}
\tightlist
\item
  Wilipedia for OLS
\end{enumerate}

\url{https://en.wikipedia.org/wiki/Ordinary_least_squares}

\begin{enumerate}
\def\labelenumi{\arabic{enumi}.}
\setcounter{enumi}{3}
\tightlist
\item
  Gradient and Jacobian
\end{enumerate}

\url{https://math.stackexchange.com/questions/1519367/difference-between-gradient-and-jacobian}

\url{https://www.youtube.com/watch?v=3xVMVT-2_t4}

\url{https://math.stackexchange.com/questions/661195/what-is-the-difference-between-the-gradient-and-the-directional-derivative}

\begin{enumerate}
\def\labelenumi{\arabic{enumi}.}
\setcounter{enumi}{4}
\tightlist
\item
  Hessian
\end{enumerate}

\url{https://en.wikipedia.org/wiki/Hessian_matrix}

\begin{enumerate}
\def\labelenumi{\arabic{enumi}.}
\setcounter{enumi}{5}
\tightlist
\item
  Observed information
\end{enumerate}

\url{https://en.wikipedia.org/wiki/Observed_information}

\begin{enumerate}
\def\labelenumi{\arabic{enumi}.}
\setcounter{enumi}{6}
\tightlist
\item
  Fisher information
\end{enumerate}

\url{https://people.missouristate.edu/songfengzheng/Teaching/MTH541/Lecture\%20notes/Fisher_info.pdf}

\begin{enumerate}
\def\labelenumi{\arabic{enumi}.}
\setcounter{enumi}{7}
\tightlist
\item
  Link function
\end{enumerate}

\url{https://en.wikipedia.org/wiki/Generalized_linear_model\#Link_function}

\url{https://stats.stackexchange.com/questions/40876/what-is-the-difference-between-a-link-function-and-a-canonical-link-function}

\bibliography{book.bib,packages.bib}

\end{document}
